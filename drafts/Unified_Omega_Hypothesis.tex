\documentclass[12pt]{article}

\usepackage{amsmath, amssymb}
\usepackage{geometry}
\usepackage{setspace}
\usepackage{hyperref}

\geometry{a4paper, margin=1in}
\setstretch{1.2}

\title{
\textbf{Unified Omega Hypothesis}\\
\large Human History as a Computation Toward the Final Information State of the Universe
}

\author{
Hiroshi Kohashiguchi \\
Independent Researcher \\
Tokyo, Japan
}

\date{}

\begin{document}

\maketitle

\begin{abstract}
This paper proposes the Unified Omega Hypothesis, a transdisciplinary framework that 
identifies a deep structural equivalence between three seemingly independent concepts: 
Chaitin’s halting probability $\Omega$ in algorithmic information theory, 
Teilhard de Chardin’s Omega Point in evolutionary theology, 
and the post-singularity cosmology articulated in contemporary AI futurism.

We reinterpret Chaitin’s $\Omega$ not merely as a mathematical constant, 
but as the total informational summation of all existential trajectories. 
By mapping individual human lives to computational programs 
and biological death to halting events, human history itself is modeled 
as an ongoing $\Omega$-computation process. 
Under this interpretation, the Omega value becomes fully determined 
only when the final conscious subject terminates—at which point 
the $\Omega$ computation completes.

Simultaneously, this terminal event is shown to be structurally identical to Teilhard’s Omega Point: 
the final convergence of all consciousness, history, and meaning. 
Within this framework, the technological singularity is reclassified not as the ultimate endpoint of evolution, 
but as a critical acceleration phase within a longer irreversible convergence toward the final informational state of the universe.

The Unified Omega Hypothesis thus reframes eschatology as an information-theoretic certainty 
rather than a supernatural intervention. It establishes a formal conceptual bridge between 
computability limits, consciousness evolution, and cosmological finality, 
suggesting that existence itself can be coherently interpreted as a computation whose completion 
corresponds to the ultimate unification of information, consciousness, and reality.
\end{abstract}


\section{Introduction}

The purpose of this paper is to propose the Unified Omega Hypothesis: 
a theoretical framework that identifies a structural unity between three concepts 
traditionally regarded as belonging to distinct intellectual domains: 
Chaitin’s $\Omega$, Teilhard de Chardin’s Omega Point, and post-singularity cosmology.

Although these ideas originated in mathematics, theology, and technological futurism respectively, 
we argue that their conceptual cores are formally isomorphic. 
Each represents a limit that can be approached but not computed or comprehended 
from within the system undergoing evolution.

The hypothesis presented here attempts to unify these perspectives by treating 
existence itself as computation, with human history corresponding to the 
progressive evaluation of an infinite $\Omega$-like summation, 
and the end of history representing the final determination of the universe’s 
informational Omega value.


\section{Background and Motivation}

\subsection{Chaitin’s $\Omega$ in Algorithmic Information Theory}

Chaitin’s halting probability $\Omega$ is defined as:

\begin{equation}
\Omega = \sum_{p \in \mathcal{P}} 2^{-|p|}
\end{equation}

where $\mathcal{P}$ is the set of all halting programs on a prefix-free universal Turing machine.

Key properties include:

\begin{itemize}
\item $\Omega$ encodes the halting information of all programs.
\item $\Omega$ is algorithmically random and uncomputable.
\item $\Omega$ becomes fully knowable only after the halting status of every possible program is determined.
\end{itemize}

\subsection{Teilhard’s Omega Point}

Teilhard’s Omega Point describes the final convergence of consciousness and complexity 
as evolution advances toward unification, producing a terminal attractor of 
meaning, purpose, and intelligence.

\subsection{Post-Singularity Cosmology}

Modern futurism suggests that the universe is moving toward maximal information density 
and computation, with technological intelligence accelerating toward 
an integrated informational cosmos.


\section{Conceptual Mapping}

Despite disciplinary differences, Chaitin’s $\Omega$, the Omega Point, 
and singularity cosmology share a common structural feature: 
each represents an ultimate limit, a final state of complete 
integration or determination.

\[
\Omega = \text{Final informational state of computation}
\]
\[
\text{Omega Point} = \text{Final unification of consciousness}
\]
\[
\text{Post-singularity universe} = \text{Final integration of intelligence}
\]


\section{Human Lives as Programs}

We propose a mapping between computation and existence:

\begin{itemize}
\item Program $\rightarrow$ Human life
\item Program generation $\rightarrow$ Birth
\item Halting $\rightarrow$ Death
\item Program length $\rightarrow$ Complexity of lived experience
\item Measure $2^{-|p|}$ $\rightarrow$ Weight of existence
\item $\Omega$ $\rightarrow$ Total informational summation of all lives
\end{itemize}

Under this interpretation, human history is the computation of $\Omega$. 
Each life contributes a measurable segment to the total informational value. 
$\Omega$ is determined only when the final conscious entity ceases to exist.


\section{Omega Point as Completion of the $\Omega$-Process}

The unified hypothesis reinterprets the Omega Point as the moment 
when the $\Omega$ computation is complete. 
Eschatology becomes information theory, consciousness evolution becomes 
progressive determination of $\Omega$, and the end of consciousness becomes 
the universal informational fixpoint.


\section{Singularity as Pre-Omega Acceleration}

The technological singularity is not the end of history but an acceleration phase 
within the $\Omega$-process. Intelligence expands, consciousness integrates, 
and computation accelerates—but $\Omega$ cannot be determined until all conscious 
processes terminate. The singularity is thus a threshold, not a completion.


\section{Ontological Implications}

This framework implies:

\begin{itemize}
\item Existence is computation.
\item Death is halting.
\item History is an $\Omega$-process.
\item The Omega Point is the final informational state of the universe.
\end{itemize}

This unifies logic, physics, theology, phenomenology, and AI futurism 
into a coherent architecture.


\section{Conclusion}

The Unified Omega Hypothesis proposes that:

\[
\Omega = \text{Omega Point} = \text{Final Informational State of the Universe}
\]

Human history is the computation of $\Omega$. 
Individual lives are programs. Death corresponds to halting. 
The singularity accelerates computation but does not complete it. 
The Omega Point is reached when the final conscious entity ceases to exist, 
fixing the universe’s total informational content.


\section*{Author Information}

\noindent
\textbf{Author:} Hiroshi Kohashiguchi \\
Independent Researcher \\
Tokyo, Japan


\section*{Acknowledgments}

The author wishes to express gratitude to the many thinkers whose work provided the 
intellectual foundations for this hypothesis, including Gregory Chaitin for the mathematical 
depth of algorithmic information theory, Pierre Teilhard de Chardin for his visionary 
articulation of the Omega Point, and the researchers and futurists who continue to expand 
our understanding of consciousness, computation, and the long-term trajectory of intelligence.

This work was motivated by a desire to bridge conceptual domains that are rarely 
considered together—mathematics, theology, AI, and eschatology—and to explore 
whether their shared structures might reveal a deeper coherence underlying human 
existence and the evolution of the universe.

Any remaining errors or speculative elements are solely the responsibility of the author.


\begin{thebibliography}{99}

% Algorithmic Information Theory
\bibitem{chaitin1975}
Chaitin, G. J. (1975).
A Theory of Program Size Formally Identical to Information Theory.
\textit{Journal of the ACM}, 22(3), 329--340.

\bibitem{chaitin1987}
Chaitin, G. J. (1987).
\textit{Algorithmic Information Theory}.
Cambridge University Press.

\bibitem{chaitin2005}
Chaitin, G. J. (2005).
\textit{Meta Math!: The Quest for Omega}.
Pantheon Books.

\bibitem{kolmogorov1965}
Kolmogorov, A. N. (1965).
Three Approaches to the Quantitative Definition of Information.
\textit{Problems of Information Transmission}, 1(1), 1--7.

% Omega Point
\bibitem{teilhard1955}
Teilhard de Chardin, P. (1955).
\textit{The Phenomenon of Man}.
Harper \& Brothers.

\bibitem{teilhard1964}
Teilhard de Chardin, P. (1964).
\textit{The Future of Man}.
Harper \& Row.

\bibitem{tipler1994}
Tipler, F. J. (1994).
\textit{The Physics of Immortality: Modern Cosmology, God and the Resurrection of the Dead}.
Doubleday.

% Singularity and Computational Cosmology
\bibitem{kurzweil2005}
Kurzweil, R. (2005).
\textit{The Singularity Is Near: When Humans Transcend Biology}.
Viking Press.

\bibitem{vinge1993}
Vinge, V. (1993).
The Coming Technological Singularity: How to Survive in the Post-Human Era.
\textit{VISION-21 Symposium}, NASA Lewis Research Center.

\bibitem{tegmark2014}
Tegmark, M. (2014).
\textit{Our Mathematical Universe: My Quest for the Ultimate Nature of Reality}.
Alfred A. Knopf.

\bibitem{lloyd2006}
Lloyd, S. (2006).
\textit{Programming the Universe: A Quantum Computer Scientist Takes on the Cosmos}.
Alfred A. Knopf.

% Interdisciplinary
\bibitem{penrose1989}
Penrose, R. (1989).
\textit{The Emperor's New Mind: Concerning Computers, Minds, and the Laws of Physics}.
Oxford University Press.

\bibitem{wolfram2002}
Wolfram, S. (2002).
\textit{A New Kind of Science}.
Wolfram Media.

\end{thebibliography}


\end{document}
