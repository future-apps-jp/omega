\documentclass[11pt,a4paper]{article}

\usepackage[utf8]{inputenc}
\usepackage{amsmath,amssymb,amsthm}
\usepackage{physics}
\usepackage{hyperref}
\usepackage{graphicx}
\usepackage{booktabs}
\usepackage{algorithm}
\usepackage{algpseudocode}

% Theorem environments
\theoremstyle{definition}
\newtheorem{definition}{Definition}[section]
\newtheorem{finding}{Finding}[section]
\newtheorem{proposition}{Proposition}[section]
\newtheorem{remark}{Remark}[section]

\title{Artificial Physics: \\
Evolutionary Emergence of Quantum Structures \\
in Resource-Constrained DSL Competition}

\author{
  Hiroshi Kohashiguchi\\
  Independent Researcher\\
  Tokyo, Japan
}

\date{December 2025}

\begin{document}

\maketitle

\begin{abstract}
We present a computational framework for understanding the emergence of 
quantum-like structures through evolutionary competition of Domain-Specific 
Languages (DSLs) under resource constraints. By simulating the competition 
between Scalar DSLs (classical operations) and Matrix DSLs (quantum-like 
operations), we demonstrate that matrix operations naturally dominate when 
selection pressure favors minimal description length. Our experiments with 
$N=100$ population size and 10 independent runs show that Matrix DSLs achieve 
100\% dominance within $2.4 \pm 0.5$ generations (95\% CI: [2.0, 2.8]), providing 
evidence for the \emph{Substrate Hypothesis}: if the universe's computational 
substrate is quantum-native, then quantum mechanics emerges as the algorithmically 
natural description. We further investigate the spontaneous emergence 
of matrix operations from scalar-only DSLs across 1000 generations with 
$N=100$ population, finding that while pattern compression creates compound 
operators, the ``conceptual leap'' to matrix operations does not occur 
spontaneously---suggesting that quantum structure may be a fundamental 
feature of the computational substrate, rather than an emergent property 
of classical computation.
\end{abstract}

%==============================================================================
\section{Introduction}
%==============================================================================

The question ``Why is the universe quantum mechanical?'' has puzzled physicists 
and philosophers since the inception of quantum theory. Previous work has 
established that quantum structure cannot be derived from classical computational 
models such as SK combinatory logic \cite{kohashiguchi2024independence} or 
reversible computation \cite{kohashiguchi2024limits}. Furthermore, 
\cite{kohashiguchi2024axioms} identified axiom A1 (state space extension / 
superposition) as the unique primitive axiom required for quantum mechanics, 
while \cite{kohashiguchi2024naturalness} demonstrated that quantum mechanics 
has minimal description length on a quantum substrate.

This work builds upon a rich tradition of \emph{digital physics}---the idea 
that the universe is fundamentally computational. Wolfram's ``A New Kind of 
Science'' \cite{wolfram2002} and subsequent physics project \cite{wolfram2020} 
explored cellular automata and hypergraphs as computational substrates. 
Lloyd's ``Programming the Universe'' \cite{lloyd2006} framed the universe as 
a quantum computer. In the artificial life community, systems like Avida 
\cite{lenski2003} and genetic programming \cite{koza1992} have demonstrated 
how complex structures can emerge through evolutionary pressure.

In this paper, we approach the foundational question from a novel perspective: 
\emph{evolutionary DSL competition}. We model physical laws as Domain-Specific 
Languages (DSLs) competing for computational resources in a ``parent universe'' 
(Localhost). The selection pressure is \emph{algorithmic naturalness}---laws that 
provide shorter descriptions (lower Kolmogorov complexity $K$) are favored.

This work provides an \emph{evolutionary complement} to our previous results: 
while \cite{kohashiguchi2024axioms, kohashiguchi2024naturalness} established 
the \emph{static} properties of quantum structure (minimality, uniqueness), 
the present paper demonstrates its \emph{dynamic} emergence through selection.

Our key contributions are:

\begin{enumerate}
    \item \textbf{Genesis-Matrix Framework}: A simulation environment for 
    DSL evolution under resource constraints.
    
    \item \textbf{Quantum Dawn Experiment}: Demonstration that Matrix DSLs 
    dominate Scalar DSLs through pure selection pressure.
    
    \item \textbf{Evolution of Operators}: Investigation of spontaneous 
    emergence of matrix operations from scalar primitives.
    
    \item \textbf{Artificial Physics}: A theoretical framework identifying 
    physical laws as optimally compressed DSLs.
\end{enumerate}

%==============================================================================
\section{Theoretical Framework}
%==============================================================================

\subsection{The Substrate Hypothesis}

\begin{definition}[Substrate Hypothesis]
The universe's computational substrate is \emph{quantum-native} ($U_Q$), 
not classical ($U_C$). Formally:
\begin{equation}
    K_{U_Q}(\text{QM}) \ll K_{U_C}(\text{QM})
\end{equation}
where $K_U(L)$ denotes the Kolmogorov complexity of physical law $L$ 
on substrate $U$.
\end{definition}

\subsection{Localhost-Container Model}

We model the ``parent universe'' as a \emph{Localhost} that allocates 
computational resources to \emph{Containers} (child universes).

\begin{definition}[Localhost]
A Localhost $\mathcal{L}$ is a tuple $(M, \{C_i\}, \mathcal{A})$ where:
\begin{itemize}
    \item $M$ is the total available memory (holographic bound)
    \item $\{C_i\}$ is the set of Containers
    \item $\mathcal{A}$ is the resource allocation function
\end{itemize}
\end{definition}

\begin{definition}[Container]
A Container $C$ is a tuple $(P, D, f)$ where:
\begin{itemize}
    \item $P$ is a program (encoding of physical law)
    \item $D$ is the DSL (vocabulary of operations)
    \item $f$ is the fitness score
\end{itemize}
\end{definition}

\begin{remark}[Physical Correspondence]
The Localhost-Container model admits natural physical interpretations:
Localhost $\leftrightarrow$ meta-universe or quantum gravity vacuum;
Container $\leftrightarrow$ pocket universe or cosmological bubble;
Memory bound $M$ $\leftrightarrow$ Bekenstein bound;
DSL $\leftrightarrow$ physical laws.
These correspondences are suggestive rather than rigorous, and serve 
primarily as conceptual motivation.
\end{remark}

\subsection{Complexity Measures}

We define operational measures for description length and execution time:

\begin{definition}[Description Length $K$]
The description length $K(P)$ of a program $P$ is the number of nodes 
in its Abstract Syntax Tree (AST). This serves as a computable proxy 
for Kolmogorov complexity.
\end{definition}

\begin{definition}[Execution Time $T$]
The execution time $T(P)$ is the wall-clock time (in milliseconds) 
required to evaluate program $P$ on the target task.
\end{definition}

\subsection{Inflation Rate and Selection}

The \emph{inflation rate} determines resource acquisition:

\begin{equation}
    V_{\text{inf}}(C) = \frac{1}{K(P) \cdot T(P)}
\end{equation}

where $K(P)$ is the description length and $T(P)$ is execution time.

\begin{remark}[Choice of Inflation Rate]
We adopt the multiplicative form $V_{\text{inf}} \propto 1/(K \cdot T)$ as the 
simplest separable model that penalizes both complexity and slowness equally. 
Alternative forms such as $1/(aK + bT)$ were considered; preliminary experiments 
suggest that the qualitative conclusions (Matrix DSL dominance) are robust to 
this choice, though a systematic sensitivity analysis is left for future work.
\end{remark}

\begin{proposition}[Selection Pressure]
Containers with lower $K$ have higher fitness and acquire more resources, 
leading to evolutionary dominance.
\end{proposition}

%==============================================================================
\section{Experimental Setup}
%==============================================================================

\subsection{Task: Graph Walk}

The experimental task is to count paths from node $0$ to node $N-1$ in 
exactly $k$ steps on a graph with adjacency matrix $A$.

The ground truth is:
\begin{equation}
    \text{answer} = (A^k)_{0, N-1}
\end{equation}

This task naturally exhibits a complexity asymmetry: scalar DSLs must 
enumerate paths individually ($K = O(N)$ or $O(k)$), while matrix DSLs 
can compute $A^k$ directly ($K = O(1)$). We expect similar asymmetries 
in other linear-algebraic tasks (e.g., Markov chain analysis, quantum 
circuit simulation), though verification is left for future work.

\subsection{DSL Species}

\begin{table}[ht]
\centering
\begin{tabular}{lll}
\toprule
Species & Operations & Description Length $K$ \\
\midrule
Scalar & ADD, SUB, MUL, IF & $O(N)$ or $O(k)$ \\
Matrix & MATMUL, MATPOW, GET & $O(1)$ \\
\bottomrule
\end{tabular}
\caption{DSL species and their description lengths for the graph walk task. 
Here $O(\cdot)$ denotes scaling with respect to AST node count.}
\label{tab:dsl}
\end{table}

\subsection{Evolution Parameters}

The main experiments (Experiment 1) used the following parameters:
\begin{itemize}
    \item Population size: $N = 100$
    \item Generations: 50
    \item Mutation rate: 15\%
    \item Crossover rate: 30\%
    \item Elite preservation: Top 5 individuals
    \item Survival rate: 20\%
    \item Graph: 5 nodes, $k=3$ steps
    \item Initial Matrix ratio: 20\%
    \item Random seeds: 10 independent runs (seeds 0--9)
\end{itemize}

Experiment 2 was divided into two conditions to ensure rigorous testing:
\begin{itemize}
    \item \textbf{Experiment 2A (Spontaneous Emergence)}: 1000 generations with 
    NO injection, to test whether matrix operations can emerge de novo 
    (5 runs, seeds 0--4).
    \item \textbf{Experiment 2B (Simulated Discovery)}: 1000 generations with 
    matrix operations injected at generation 300, to observe post-discovery 
    dynamics (5 runs, seeds 0--4).
    \item Population size: $N = 100$ for both conditions.
\end{itemize}

Additional robustness experiments varied graph size ($N \in \{5, 8, 10\}$) 
and initial Matrix ratio ($\{5\%, 10\%, 20\%, 50\%\}$).

%==============================================================================
\section{Results}
%==============================================================================

\subsection{Experiment 1: The Quantum Dawn}

\begin{finding}[Matrix Dominance]
In the graph walk task with $N=100$ population and 10 independent runs, 
Matrix DSL achieves 100\% dominance within $2.4 \pm 0.5$ generations 
(95\% CI: $[2.0, 2.8]$) when starting from 20\% initial ratio.
\end{finding}

\textbf{Experimental Evidence:}

\begin{table}[ht]
\centering
\begin{tabular}{ccc}
\toprule
Run & Convergence (Gen) & Final Matrix (\%) \\
\midrule
1 & 3 & 100 \\
2 & 3 & 100 \\
3 & 2 & 100 \\
4 & 3 & 100 \\
5 & 2 & 100 \\
6 & 3 & 100 \\
7 & 2 & 100 \\
8 & 2 & 100 \\
9 & 2 & 100 \\
10 & 2 & 100 \\
\midrule
Mean & 2.4 & 100 \\
SD & 0.52 & -- \\
95\% CI & {[}2.0, 2.8{]} & -- \\
\bottomrule
\end{tabular}
\caption{Experiment 1: Matrix DSL dominance results across 10 independent 
runs with population size $N = 100$ and 50 generations.}
\label{tab:exp1}
\end{table}

The convergence is remarkably fast and highly reproducible---all 10 runs 
achieved 100\% Matrix dominance within 2--3 generations. The 95\% confidence 
interval for convergence generation is narrow ($[2.0, 2.8]$), indicating 
strong statistical reliability. Additional robustness experiments confirmed:
\begin{itemize}
    \item \textbf{Scaling}: Convergence occurred for all tested graph 
    sizes ($N = 5, 8, 10$).
    \item \textbf{Initial ratio threshold}: With only 5\% initial Matrix 
    ratio, convergence failed; with 10\% or higher, convergence succeeded.
\end{itemize}

\subsection{Experiment 2: Evolution of Operators}

\subsubsection{Experiment 2A: Spontaneous Emergence Test}

\begin{finding}[No Spontaneous Emergence]
In Experiment 2A with $N=100$ population and 1000 generations (5 runs, 
NO injection), matrix operations did \textbf{not emerge spontaneously 
in any run} (0/5). Pattern compression created 20 new compound scalar 
operators per run, but none were matrix-like.
\end{finding}

\begin{table}[ht]
\centering
\begin{tabular}{cccc}
\toprule
Run & Final Operators & Inventions & Matrix Emerged \\
\midrule
1 & 25 & 20 & No \\
2 & 25 & 20 & No \\
3 & 25 & 20 & No \\
4 & 25 & 20 & No \\
5 & 25 & 20 & No \\
\bottomrule
\end{tabular}
\caption{Experiment 2A: Spontaneous emergence test (NO injection, 
1000 generations). Matrix operations did not emerge in any of the 5 runs.}
\label{tab:exp2a}
\end{table}

This is the central evidence for our claim: despite 1000 generations of 
evolution and 20 invented compound operators, the ``conceptual leap'' 
to matrix operations never occurred spontaneously.

\subsubsection{Experiment 2B: Simulated Discovery Test}

\begin{finding}[Post-Discovery Dynamics]
In Experiment 2B, matrix operations were injected at generation 300 
to simulate their ``discovery.'' Notably, scalar evolution had already 
converged to a local optimum (Fitness $\approx 400$) by this point, 
so no immediate fitness jump was observed.
\end{finding}

\begin{table}[ht]
\centering
\begin{tabular}{ccccc}
\toprule
Run & Pre-Injection & Post-Injection & Jump & Final Ops \\
\midrule
1 & 400.0 & 400.0 & +0.0 & 26 \\
2 & 400.0 & 400.0 & +0.0 & 26 \\
3 & 400.0 & 400.0 & +0.0 & 26 \\
4 & 400.0 & 400.0 & +0.0 & 26 \\
5 & 400.0 & 400.0 & +0.0 & 26 \\
\bottomrule
\end{tabular}
\caption{Experiment 2B: Simulated discovery test (injection at Gen 300). 
No fitness jump because scalar DSL had already found a task-specific solution.}
\label{tab:exp2b}
\end{table}

\textbf{Interpretation:} The absence of fitness jump in Experiment 2B 
is \emph{expected} for this specific task: scalar DSLs can find 
``lucky'' solutions that produce the correct answer for fixed $(N, k)$. 
The crucial distinction is:

\begin{itemize}
    \item \textbf{Scalar solutions}: Task-specific, no generalization. 
    A program that works for $(N=5, k=3)$ will fail for $(N=10, k=5)$.
    \item \textbf{Matrix solutions}: Structurally general. The program 
    $\texttt{GET(MATPOW(A, k), 0, N-1)}$ works for \emph{any} $(N, k)$.
\end{itemize}

This highlights a limitation of fitness-based selection on fixed tasks: 
it cannot distinguish between ``memorization'' (scalar) and ``understanding'' 
(matrix). The true advantage of matrix operations manifests in 
\emph{generalization} across task variants.

%==============================================================================
\section{Discussion}
%==============================================================================

\subsection{Implications for the Substrate Hypothesis}

Our results provide evidence consistent with the Substrate Hypothesis:

\begin{enumerate}
    \item \textbf{Matrix DSL dominance}: When both scalar and matrix 
    operations are available, matrix operations dominate due to their 
    superior compression. This is consistent with quantum mechanics being 
    ``natural'' on a quantum substrate.
    
    \item \textbf{Emergence difficulty}: In Experiment 2A, matrix 
    operations did not emerge spontaneously in any of 5 runs over 
    1000 generations (0/5), despite 20 compound operators being invented. 
    This suggests matrix structure may need to be ``built into'' the substrate.
    
    \item \textbf{Rapid takeover}: Once matrix operations exist, they 
    achieve 100\% dominance within 2--3 generations across all 10 runs 
    (95\% CI: $[2.0, 2.8]$)---consistent with quantum mechanics being the 
    minimal description on a quantum-native substrate.
\end{enumerate}

\subsection{Why Is the Universe Quantum?}

Our framework suggests a conditional answer:

\begin{quote}
\emph{If} the universe's computational substrate is quantum-native, 
\emph{then} quantum mechanics provides the shortest description 
(minimal $K$), and evolutionary selection for minimal $K$ would 
naturally lead to quantum-like physical laws.
\end{quote}

This reframes the question from ``Why quantum?'' to ``What properties 
must the substrate have for quantum mechanics to be minimal?''

\subsection{Artificial Physics}

We define \emph{Artificial Physics} as the study of physical laws as 
evolutionarily selected DSLs:

\begin{definition}[Artificial Physics]
Physical laws are DSLs that have been selected by evolutionary pressure 
to minimize description length $K$ on the underlying computational substrate.
\end{definition}

This perspective reframes fundamental physics questions:
\begin{itemize}
    \item ``Why quantum mechanics?'' $\to$ ``What substrate makes QM minimal?''
    \item ``What is a physical law?'' $\to$ ``An optimally compressed DSL''
    \item ``Why these constants?'' $\to$ ``They minimize $K$ on $U_Q$''
\end{itemize}

\subsection{Limitations and Future Work}

Several limitations of this study should be noted:

\begin{enumerate}
    \item \textbf{Single task}: Our experiments focus on the graph walk 
    task. While we expect similar results for other linear-algebraic tasks, 
    this remains to be verified.
    
    \item \textbf{Idealized model}: The Localhost-Container model abstracts 
    away many physical complexities (noise, decoherence, imperfect observation).
    
    \item \textbf{Limited DSL space}: The DSL search space is constrained 
    by our chosen primitives; richer primitive sets might yield different 
    emergence dynamics.
    
    \item \textbf{Fixed-task limitation}: Experiment 2B showed no fitness 
    jump because scalar DSLs can find task-specific solutions. Future 
    work should use \emph{meta-fitness} that rewards generalization 
    across task variants $(N, k)$.
\end{enumerate}

Future work will address these limitations by: (1) testing additional 
tasks (Markov chains, quantum circuits), (2) exploring richer mutation 
operators that might enable genuine emergence, and (3) formalizing the 
connection between the Localhost-Container model and established physics 
(holographic principle, cosmological natural selection).

%==============================================================================
\section{Conclusion}
%==============================================================================

We have demonstrated through simulation that:

\begin{enumerate}
    \item Matrix DSLs (quantum-like structures) dominate Scalar DSLs 
    (classical structures) under selection pressure for minimal 
    description length.
    
    \item The dominance is rapid ($2.4 \pm 0.5$ generations; 95\% CI: 
    $[2.0, 2.8]$) and robust across different graph sizes, initial 
    conditions, and 10 independent runs.
    
    \item Spontaneous emergence of matrix operations from scalar 
    primitives did not occur in any of 5 runs over 1000 generations 
    (Experiment 2A: 0/5), despite 20 compound operators being invented 
    per run. This suggests that quantum structure may be fundamental 
    rather than emergent.
\end{enumerate}

These results are consistent with the \textbf{Substrate Hypothesis}: 
if the universe's computational substrate is quantum-native, then quantum 
mechanics would be the ``natural'' physical law on this substrate.

The \textbf{Artificial Physics} framework provides a new lens for 
understanding fundamental physics: physical laws as evolutionarily 
selected, algorithmically optimal descriptions on the computational 
substrate of reality.

%==============================================================================
\section*{Acknowledgments}
%==============================================================================

The author thanks Gregory Chaitin for foundational work on algorithmic 
information theory, and Stephen Wolfram for pioneering the computational 
universe paradigm. All implementations are released as open-source software 
under the MIT License at \url{https://github.com/future-apps-jp/omega/}.

%==============================================================================
\begin{thebibliography}{99}

\bibitem{kohashiguchi2024independence}
H. Kohashiguchi,
``On the Independence of Quantum Structure from SK Combinatory Logic,''
PhilArchive, 2025.

\bibitem{kohashiguchi2024limits}
H. Kohashiguchi,
``Computational Limits of Deriving Quantum Structure from Reversible Logic,''
PhilArchive, 2025.

\bibitem{kohashiguchi2024axioms}
H. Kohashiguchi,
``Minimal Axioms for Quantum Structure,''
PhilArchive, 2025.

\bibitem{kohashiguchi2024naturalness}
H. Kohashiguchi,
``Algorithmic Naturalness on a Quantum Substrate,''
PhilArchive, 2025.

\bibitem{chaitin1975}
G. J. Chaitin,
``A theory of program size formally identical to information theory,''
\emph{J. ACM}, vol. 22, no. 3, pp. 329--340, 1975.

\bibitem{wolfram2002}
S. Wolfram,
\emph{A New Kind of Science},
Wolfram Media, 2002.

\bibitem{wolfram2020}
S. Wolfram,
``A Class of Models with the Potential to Represent Fundamental Physics,''
\emph{Complex Systems}, vol. 29, pp. 107--536, 2020.

\bibitem{lloyd2006}
S. Lloyd,
\emph{Programming the Universe: A Quantum Computer Scientist Takes on the Cosmos},
Knopf, 2006.

\bibitem{lenski2003}
R. E. Lenski, C. Ofria, R. T. Pennock, and C. Adami,
``The evolutionary origin of complex features,''
\emph{Nature}, vol. 423, pp. 139--144, 2003.

\bibitem{koza1992}
J. R. Koza,
\emph{Genetic Programming: On the Programming of Computers by Means of Natural Selection},
MIT Press, 1992.

\end{thebibliography}

\end{document}
